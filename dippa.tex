\documentclass[a4paper]{article}

% Encoodaus, joka sopii suomenkielellä (esim. ä ja ö)
\usepackage[utf8]{inputenc}
\usepackage[T1]{fontenc}

% Suomenkielinen tavutus
\usepackage[finnish]{babel}

% Viitteet
\usepackage{natbib}

\usepackage{alltt}

% Otsikkojen päätteetön fontti
% \usepackage{sectsty}
% \allsectionsfont{\sffamily\large}

% Viitteiden merkit
\bibpunct{(}{)}{;}{a}{,}{,}

\usepackage{titlesec}
\titleformat{\subsection}{\normalsize\bfseries}{}{0em}{}{}
\titlespacing{\subsection}{0pt}{*1.5}{*1.0}

\begin{document}

\title{\small Yrityksen rahoitus \\ \huge Tenttivastaukset}
\date{25.2.2012}
\author{Mikko Koski \\ mikko.koski@aalto.fi \\ 66467F}
\maketitle

\normalsize
\setlength{\parindent}{0cm}

\usepackage{titlesec}

\titlespacing{\paragraph}{%
  0pt}{%              left margin
  0.5\baselineskip}{% space before (vertical)
  1em}%               space after (horizontal)

% \section{Tentti 2009-06}

\subsection{Diskonttokorko (engl. discount rate)}



\subsection{Beta (termi ja laskeminen)}

Beta kuvaa osakkeen systemaattista riskiä. Beta kuvaa sitä, miten paljon markkinoiden suhtanteet vaikuttavat osakkeen arvoon. Korkean betan osakkeiden arvo laskevat/nousevat huomattavasti markkinoiden mukana, kun taas matalan betan osakkeet pitävät hyvin arvonsa markkinoista huolimatta. Beta voi olla myös negatiivinen, jolloin osakkeen arvo muuttuu päinvastoin kuin markkinoiden arvo.



\subsection{Velan suhde omaan pääomaan (engl. debt-equity ratio)}

\subsection{Riskittömän tuoton korko (engl. risk-free rate)}

\subsection{Markkinaportfolion riskipreemio (engl. expected risk premium on the market portfolio)}

\subsection{Projektin tuottovaatimus (laskeminen) (engl. required return for the project)}

\subsection{Osakkeiden tuottovaatimus (laskeminen) (required return on the shares)}

\subsection{Option hintaan vaikuttavat tekijät (essee) (engl. factors affecting option prices)}

\subsection{Put-call pariteetti (essee) (engl. put-call-parity)}

\subsection{Täydelliset markkinat (engl. perfect capital markets)}

\subsection{Omavaraisuusaste (engl. equity asset ratio)}

\subsection{Omavasaisuusasteen vaikutus yrityksen arvoon}

\subsection{Suuret institutionaaliset sijoittajat suosivat monikansallisia monitoimialayrityksiä, koska sijoitukset niihin ovat vain yhdellä toimialalla toimivia yrityksiä riskittömämpiä. (Totta vai tarua?)}

\subsection{Keskimääräinen pääoman kustannus (WACC) (engl. weighted average cost of capital)}

\subsection{Nettonykyarvo (Net present value, NPV)}

NPV:llä voidaan laskea yksittäisen investointiprojektin kannattavuutta. Silloin lasketaan itse asiassa arvio \textbf{investoinnin aiheuttamasta muutoksesta yrityksen liiketoiminnan arvossa}. 

\subsection{Kun projektin kannattavuutta arvioidaan NPV:n avulla, tulisi aina käyttää diskonttokorkona yrityksen keskimääräistä pääoman kustannusta (WACC) (Totta/tarua?)}

\subsection{Modiglianin ja Millerin (MM) teoriat I ja II (taustat, teoriat, johtopäätökset)}

\subsection{Markkinoiden heikon tehokkuuden olettamus (weak form market efficiency)}

\subsection{Huomaat, että yrityksen johtajat tekevät suuria voittoja sijoittamalla oman yrityksensä osakkeisiin. Tämä on vastoin markkinoiden heikon tehokkuuden olettamusta (totta/tarua)}

\subsection{Mitä enemmän osakkeen tuotto vaihtelee, sitä korkeampaa tuottoa sijoittajat odottavat (totta/tarua)}

\subsection{Osto-optio (Call)}

\subsection{Myyntioptio (Put)}

\subsection{Osto-option myyminen on tuotoltaan identtinen myyntioption ostamisen kanssa}

\subsection{Oman pääoman kustannus (cost of equity)}

\subsection{Osakkeiden takaisinosto (repurchase stock)}

\subsection{Markkinaportfolion riskipreemio (risk premium on market portfolio)}

\subsection{Markkinaportfolion keskihajonta (market portfolio standerd deviation)}

\subsection{Markkinaportfolion riskipreemio on 8\% ja keskihajonta 22\%. Mikä on sellaisen portfolion riskipreemio, jonka sijoituksista 25\% on kiinni Onkia Oy:n osakkeissa ja 75\% Edll Oy:n osakkeissa? Onkian beta on 1,1 ja Edll:n 1,25}

\subsection{Vapaa kassavirta (Free cash flow FCF)}

Yrityksen liiketoiminta synnyttää vapaata rahavirtaa.

Vapaa kassavirta lasketaan

\[
\begin{array}{lcl}
Free\ Cash\ Flow & = & Unlevered\ Net\ Income \\
 & & + Deprication \\
 & & - Capital\ Expenditures \\
 & & - Increase\ in\ Net\ Working\ Capital \\
\end{array}
\]

\[Unlevered\ Net\ Income = EBIT - Intrests - Taxes\]


Missä: Deprication = verot, Capital expeditures = investoinni, Increase in Net Working Capital = Nettopääoman kasvu

Vapaa rahavirta määrittää liiketoiminnan arvon.

\[
V_0 = \frac{FCF_1}{1 + r_{wacc}} + \frac{FCF_2}{(1 + r_{wacc})^2} + \dots + \frac{FCF_N}{(1 + r)^N} + \frac{V_N}{(1 + r_{wacc})^N}
\]

\subsection{Nettopääoman (Net working capital)}

Nettopääoman kasvu

\[ \Delta NWC_t = NWC_t - NWC_{t-1} \]

\[
\begin{array}{lcl}
Net\ Working\ Capital & = & Current\ Assets + Current\ Liabilities \\
 & = & Cash + Inventory + Receivables - Payables \\
\end{array}
\]

Missä: Inventory = Vaihto-omaisuus, Receivables = Lyhytaikaiset saamiset, Payables = Lyhytaikaiset velat

\subsection{Marginaalinen veroaste (corporate marginal tax rate)}

\subsection{APV-menetelmä (Adjusted Present Value method)}

\subsection{Tase (Balance sheet)}

\subsection{Acquisition?}

% \section{Tentti 2009_03}

\subsection{Odotettu tuotto (expected return)}

Odotettu tuotto voidaan laskea eri tuotto-odotusten ja niiden toteutumisen todennäköisyyksien tulojen summasta.

Esimerkki: Yrityksen arvo on 120e. Vuoden päästä arvo on 100e 40\% tn:llä ja 170e 60\% tn:llä

Tästä saadaan: Vuoden päästä tuotto on 40\% tn:llä -16.67\% ja 70.59\% tn:llä 60\%

Odotettu tuotto (expected (mean) return):

\[
E[R] = \sum_R{p_R \times R}
\]

Missä:

$p_R$ \quad Todennäköisyys
$R$ \quad Todennäköisyyttä vastaava tuotto-odotus

Tästä saadaan:

\[
E[R] = 0.4 \times -16.67\% + 0.6 \times 70.59\% = 49.02%
\]

Odotetun tuoton avulla voidaan laskea varianssi ja keskihajonta eli volatiliteetti.

\subsection{Volatiliteetti}

Varianssin laskeminen odotetun keskimääräisen tuoton avulla.

Esimerkki: Yrityksen arvo on 120e. Vuoden päästä arvo on 100e 40\% tn:llä ja 170e 60\% tn:llä (, jolloin odotettu tuotto on 49.02%)

Tästä saadaan: Vuoden päästä tuotto on 40\% tn:llä -16.67\% ja 70.59\% tn:llä 60\%

\[
Var(R) = E[(R - E[R])^2] = \sum_R{p_R \times (R - E[R])^2}
\]

Kaavassa $R - E[R]$ on poikkeama keskimääräisestä tuotosta

Esimerkkiyritykselle saadaan tällöin

\[
Var(R) = 40\% \times (49.02\% + 16.67\%)^2 + 60\% \times (49.02\% - 70.59\%)^2 = 0.03059
\]

Volatiliteetti, eli keskihajonta (standard deviation) on varianssin neliö:

Volatility = Standard Deviation = $\sqrt{Variance}$

Tällöin voidaan laskea esimerkille volatiliteetti

\[
SD(R) = \sqrt{Var(R)} = \sqrt{0.03059} = 17.49\%
\]

\subsection{Korrelaatio (correlation)}



\subsection{Olet investoinut Coda rahastoon, jolla on 11\% odotettu tuotto ja 20\% volatiliteetti. Pohdit pitäisikö sinun lisätä Venture rahastoa omaan sijoitusportfolioosi. Venture rahaston odotettu tuotto on 19\%, volatiliteetti 60\% ja korrelaatio 0,25 Cofa rahaston kanssa. Laske annettujen tiedon perusteella kannattaako Venture rahastoa lisätä omaan sijoitusportfolioosi. Riskitön korkokanta on 3\%}

\subsection{Pääomarakenteen vaikutus yrityksen arvoon (täydellisillä markkinoilla) (how capital structure affects the value)}

\subsection{Tyypillisimmät epätäydellisyydet ja vaikutukset (suorat ja epäsuorat) pääomarakenteeseen ja arvoon
(the most common imperfections)}

\subsection{Laske eurooppalaisen osto-option hinta kun lunastushinta eli toteutushinta on X (The price of European call option having an exercie price of X)}

\subsection{Yhtiövero (corporate tax rate)}

\subsection{Velaton pääoman kustannus (unlevered cost of capital)}

\subsection{Yrityksen arvo ilman velkaa (value without leverage)}

\subsection{Yrityksen arvo velan kanssa (value with leverage)}

% \section{Tentti 2007_03}

\subsection{Yrityksen arvo tänään (Company's value today)}

\subsection{Payout ratio}

\subsection{Security market line}

\subsection{Markkinaportfolio}

\subsection{Markkinariski}

\subsection{Efficient portfolios}

Portfolio, jonka yksittäisten osakkeiden yhtiökohtaiset riskit on minimoitu hajauttamalla. Tehokas portfolio sisältää vain markkinoiden systemaattista riskiä.

Esimerkki:

Markkinaportfolion kehitys: 47\% hyvillä markkinoilla, -25\% huonoilla
Osakkeen kehitys: 40\% hyvillä markkinoilla, -20\% huonoilla.

Mikä on osakkeen beta?

$\Delta Mkt = 47\% - (-25\%) = 72\% $

$\Delta Stock = 40\% - (-20\%) = 60\%$

$\beta Stock = \frac{60\%}{72\%} = 0.833 $

\subsection{Unique risk}

\subsection{Risk-free interest rate}

\subsection{Standard deviation of return}

\subsection{Expected return}

\subsection{Pääoman rakenteen vaikutuksista täydellisillä ja epätäydellisillä markkinoilla}

\subsection{Hinta joko nousee tai laskee (miten vaikuttaa laskuissa?)}

% \section{Tentti 2005_05}

\subsection{MM I proposition}

\subsection{OTC market}

\subsection{Tax shield}

\subsection{WACC}

Pääoman painotettu keskimääräinen kustannus.

Painoitettu tarkoittaa sitä, että otetaan huomioon velka/omapääoma suhde.

\[
r_{WACC} = \frac{E}{E + D} * r_E + \frac{D}{E + D} * r_D * (1 - \tau_c)
\]

Missä: 

E = equity, D = debt

$r_E$ = oman pääoman kustannus = \textbf{CAPM}, Capital Asset Pricing Model

$r_D$ = vieraan pääoman kustannus = Luottoriski

Huom! Laskuissa annetaan usein velan suhde omaan pääomaan (debt/equity ratio). Tästä voidaan laskea:

\[
\frac{E}{E + D} = \frac{1}{1 +r}
\]

\[
\frac{D}{E + D} = \frac{r}{1 +r}
\]

Jos esim D/E ratio on 0.6, saadaan

\[
\frac{E}{E + D} = \frac{1}{1 + 0.6} = 0.625
\]

\[
\frac{D}{E + D} = \frac{r}{1 +r} = 0.375
\]

\subsection{Unique risk}

\subsection{Semi-strong market efficiency}

\subsection{Pääoman markkina-arvo (market value of outstanding equity)}

Arvo, jonka markkinat ovat valmiita maksamaan yrityksen omistamasta pääomasta (tehtaat, tuotantolaitteet ym.)

Suhde liiketoiminnan arvoon (Enterprise Value, EV)

\[Enterprise Value = Market Value Of Equity + Debt - Cash\]

\subsection{Odotettu tuotto (expected rate of return)}

\subsection{Option toteutushinta (exercise price)}

\subsection{Option erääntymispäivä (maturity date)}

\subsection{Tuottokäytä (payoff graph)}

\subsection{Voittokäyrä (profit graph)}

\subsection{Jos kohdeosakkeen beta on positiivinen, onko tämän portfolion beta positiivinen vai negatiivinen?}

\subsection{Ulosmaksusuhde (pay-out ratio)}

\subsection{Mitä suurempi yrityksen ulosmaksusuhde on, sitä parempi se on yrityksen osakkeenomistajille (totta/tarua?)}

\subsection{Varrantti (Warrant)}

\subsection{Varrantti antaa omistajilleen mahdollisuuden vaihtaa jokaisen velkakirjan ennalta määrättyyn määrään osakkeita (totta/tarua?)}

\subsection{Salkunhoitaja Miettinen on hankkimassa uusia osakkeita salkkuunsa. Hän analysoi GWB yritystä ja antaa sille arvosanan 3. Markkinoiden yksimielinen arvio taas on 2. Näin ollen Miettisen ei kannata ostaa yrityksen osakkeita.}

\subsection{Vaikutus osto-option arvoon?}
a) Osakkeen hinta nousee (stock price rises)
b) Korko nousee (interest rate rises)
c) Erääntymisaika kasvaa (time to maturity becomes larger)
d) Osakkeen volatiliteetti kasvaa (stock's volatility increases)

\subsection{Valtion obligaatio (Treasure bond)}

\subsection{Obligaation korko (yield)}

\subsection{Obligaation duraatio (duration)}

\subsection{Obligaation volatiliteetti}

% Muut

\subsection{Yhden hinnan laki}

Arbitraasin (riskittömän pikavoiton) mahdollisuus on lyhytikäinen.
Esimerkki: kullan hinta on sama Lontoossa ja New Yorkissa

\subsection{Nykyarvo (Present value)}

\[
PV = \frac{C}{(1 + r)^n}
\]

Missä: C = nykyarvo, r = tuottovaatimus

Useampana ajanhetkenä toteutuvan rahavirran nykyarvo:

\[
PV = \frac{C_1}{(1 + r)} + \frac{C_1}{(1 + r)^2} + \dots + \frac{C_N}{(1 + r)^N} = \sum_{n=1}^{N} \frac{C_n}{(1 + r)^n}
\]

Ikuinen kasvava maksusuoritus

\[
PV (growing perpetuity) = \frac{C}{r - g}
\]

Missä: g = kasvuprosentti

\subsection{Kuponkimaksu (coupons)}

Joukkovelkakirjojen korkomaksu. $Kuponkikorko = Kuponkimaksu / Nimellisarvo$

\subsection{Laina-ajan päättymispäivä (maturity date)}

\subsection{Velkakirjan nimellisarvo (face value)}

\subsection{Oman pääoman tuottovaatimus}

\subsection{P/E-luku}

\subsection{CAPM, Capital Asset Pricing Model}

\[
\begin{array}{lcl}
E[R_i] & = & Risk-Free\ Interest\ Rate + Risk\ Premium \\
 & = & r_f + \beta_i \times ( E [R_{Mkt} - r_f] )
\end{array}
\]

Missä:

$\beta$ = Tarkasteltavan sijoituskohteen (i) systemaattista riskiä kuvaava suure

$( E [R_{Mkt} - r_f] )$ = Markkinaportfolion riskipreemio

\end{document}